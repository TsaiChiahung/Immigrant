
% The first version was presented at 2018 APSA, 8/30/2018.

\documentclass[12pt]{article}
\usepackage{epsf,graphicx,psfrag}
\usepackage{lscape}
\usepackage{natbib}
\usepackage{fullpage}
\usepackage{setspace}
\usepackage[top=1.2in, bottom=1.2in, right=1in, left=1in]{geometry}
\usepackage{amsmath}
\usepackage{amssymb}
\usepackage{cases}
\usepackage{empheq}
\usepackage{dcolumn}
\usepackage{color}
\usepackage{scalefnt}
\usepackage{booktabs}
\usepackage{longtable}
\usepackage{multirow}
\usepackage{grffile}
\usepackage{bm}
\usepackage{subfig}
\usepackage[hidelinks]{hyperref}    % Link to reference and footnote
\usepackage{wrapfig}
\usepackage{rotating}
\usepackage{tabularx}
\usepackage{colortbl}
\usepackage{hhline}
\usepackage{caption, fixltx2e} % Add notes under table
\usepackage[flushleft]{threeparttable} % Add notes under table

\def\citeapos#1{\citeauthor{#1}'s (\citeyear{#1})}

\newcommand{\bpm}{\begin{pmatrix}}
\newcommand{\epm}{\end{pmatrix}}
\newcommand{\bfa}{{\bf A }}
\newcommand{\bfx}{{\bf X }}
\renewcommand{\baselinestretch}{2}

\newcommand{\bbm}{\begin{bmatrix}}
\newcommand{\ebm}{\end{bmatrix}}

\newcommand{\minitab}[2][l]{\begin{tabular}{#1}#2\end{tabular}}


% DEFINE A NEW COLOR
\definecolor{Gray}{gray}{0.9}



\title{\renewcommand{\thefootnote}{\fnsymbol{footnote}}
Different Immigrants, Same Attitudes? \\ A Bayesian Bivariate Ordered Probit Analysis of \\ Public Opinion in Taiwan\footnote{Paper prepared for presentation at the Annual Meeting of the American Political Science Association in Boston, MA, USA, August 30-September 2, 2018. This research is in a preliminary stage. Comments and suggestions are welcome.}
}


\author{
Tsung-han Tsai\footnote{Associate Professor, Department of Political Science, and Research Fellow, Election Study Center and Taiwan Institute for Governance and Communication Research, National Chengchi University, Taipei, Taiwan R.O.C. (Email: thtsai@nccu.edu.tw).}
\and
Chia-hung Tsai\footnote{Research Fellow, Election Study Center, Graduate Institute of East Asian Studies, and Taiwan Institute for Governance and Communication Research, National Chengchi University, Taipei, Taiwan R.O.C. (Email: tsaich@nccu.edu.tw).}
\and
Chi Huang\footnote{University Chaired Professor, Department of Political Science, Election Study Center, and Taiwan Institute for Governance and Communication Research, National Chengchi University, Taipei, Taiwan R.O.C. (Email: chihuang@nccu.edu.tw).}
}


\date{
This Draft: \today
}        % Activate to display a given date or no date



\begin{document}
\doublespace

\pagenumbering{gobble}

\maketitle

\thispagestyle{empty}



\begin{abstract}
Previous studies of public opinion on immigration focus on economically developed countries such as the United States and Western European countries and implicitly refer immigrants to blue-collar laborers. In this article, we extend the discussion to a developing country---Taiwan---and investigate public attitudes toward immigrants with different occupations. According to the results of survey data analysis, we find different economic factors for anti-immigration attitudes toward foreign professionals and foreign laborers. Furthermore, cultural tolerance is consistently correlated to pro-immigration attitudes, including both foreign professionals and foreign laborers. The results imply that, although Taiwanese people have group-specific stereotypes in mind, economic factors are the source of different attitudes toward immigrants.
\end{abstract}



\textbf{keywords:} immigration, Taiwan Politics, Bayesian methods, bivariate ordered probit models



\newpage

\pagenumbering{arabic}

\setcounter{page}{1}



\section{Introduction}


In the age of globalization, population movements across national boundaries has evidently accelerated due to economic and political reasons. The increase in international migration inevitably brings about debates on immigration issues in the receiving countries. In democratic countries, whether the public supports or opposes granting legal status and work permits to immigrants plays an important role in shaping immigration policy. Are these attitudes different when the immigrants refer to different groups in terms of occupational skills? If so, is the public more supportive of an influx of foreign professionals than foreign laborers or vise versa, and what are the factors contributing to these differences? 


A great amount of research has examined the determinants of individual attitudes toward immigration, including attitudes toward immigrants \citep[e.g.,][]{EspenshadeCalhoun1993, EspenshadeHempstead1996} and toward immigration policy \citep[e.g.,][]{ChandlerTsai2001, Citrinetal1997}. In the literature, economic competition and cultural threat are commonly analyzed sources of anti-immigrant sentiments. The former, based on the assumption of labor-market competition, hypothesizes that individual workers will oppose immigration of workers with similar skills to their own due to the fear of competing for material resources \citep{Mayda2006, ScheveSlaughter2001}. The latter postulates that native people will be against immigration because of symbolic prejudice toward specific immigrant groups \citep{LeeFiske2006} and the perceived intergroup threat on cultural unity and/or national identity \citep{Blumer1958, BurnsGimpel2000, Quillian1995}. Although both theories have been found supported by empirical evidence, more work is needed to identify the causal mechanisms and their effects on attitude toward immigration \citep{CeobanuEscandell2010, HainmuellerHopkins2014}.


The economic and cultural concerns are applied to explaining two broad types of attitudes on immigration issues: opinion on immigrants and views on immigration policy \citep{CeobanuEscandell2010}. Regarding public attitudes toward immigrants, much of the literature does not consider the differences of skill levels among immigrants and presumes the referred immigrants are low-skilled, blue-collar workers. There are at least two reasons to differentiate between different types of immigrants. First, the implicit presumption of immigrants as blue-collar workers is highly unreasonable since international migration involves not only blue-collar workers but also white-collar ones. Second, it has been shown that, however, the differentiation between skill levels of immigrants is appropriate to test alternative theoretical arguments that explain the source of negative sentiments toward immigration \citep{HainmuellerHiscox2007}. 


To answer the above questions, we follow the research that examines public attitudes toward immigrants with different skill levels. In this article, we investigate public attitudes toward two categories of immigrants---professionals and laborers. By differentiating between professionals and laborers based on occupation skills, we are able to test both economic and cultural explanations to public support for/opposition to immigration. To test competing explanations, we apply a Bayesian bivariate ordered probit model to analyzing a nationally representative sample of 1,966 respondents in Taiwan. According to the data analysis, we find that, first, different economic factors are related to attitudes toward immigrant professionals and laborers. Second, people's tolerance of cultural diversity is consistently associated with attitudes toward immigrants regardless of immigrants' occupation.


This article has two primary contributions. First, substantively, this article contributes to our understanding of how natives view immigrants across occupation in Taiwan, a economically developing country. The evidence shows that both economic and cultural factors are related to public attitudes toward immigration but in a different way. Second, methodologically, the Bayesian bivariate ordered probit model successfully describes the positive correlation between the two response variables. The explicit modeling of the association between the two response variables potentially reflects the data generating process, which in turn possibly leads to more correct inferences.


The remainder of this article proceeds as follows. Section 2 reviews the extant literature on public opinion toward immigration. We then develop theoretical arguments that explain the effects of economic and cultural concerns on attitudes toward immigrants across occupations. Section 3 discusses the case of Taiwan and provides an illustration of the data and measurements. The statistical model employed in the empirical analysis and the results of the data analysis are displayed in Section 4. Section 5 concludes the article.



\section{Public Attitudes toward Immigration}


In the age of globalization and open-economy countries, international migration has undoubtedly accelerated and simultaneously brought about the discussions on immigration issues in host countries. One of the most important issues is the extent of immigration restrictions, which is shaped by public evaluations of the economic and social impacts of immigration on its own countries. In the following, we start with a review on the literature on public opinion toward immigration. Then we demonstrate public support for citizenship granted to immigrants induced by economic and cultural concerns. Finally, we provide several testable hypotheses appropriate for countries worldwide in general and for the particular case discussed in the third section.



\subsection{Economic and Cultural Concerns about Immigration}


As international migration has gradually increased in the past two decades, both international migrants and native population have encountered numerous challenges. The immigrants leave their home countries due to economic or political reasons for some places where they seek for a better life. \textcolor{red}{Well-educated immigrants are willing to use their expertise in their professions in different countries. Unfortunately, many countries and employers have concerns about the newcomers' language skills, educational backgrounds, and lower wages compared with the native-born.} In these new lands, however, they face certain issues such as immigration restriction and immigration attitudes of the natives. Regarding immigration restriction, immigration policies involve regulations concerning entry to, residence in, and the citizenship of a country and vary around world, ranging from allowing most types of migration to allowing no migration at all. Generally speaking, the less restrictive the regulations, the more likely the influx of immigrants. 


Concerning immigration attitudes of the natives, the native-born residents may have two contradicting attitudes about immigration. The natives, on the one hand, have an open mind on immigration due to the labor shortage, especially in developed countries where the population is gradually aging. The native people, on the other hand, are concerned with the economic impact of immigration such as native wages and job displacement \citep{ScheveSlaughter2001}, and/or social impact of immigration such as cultural homogeneity \citep{Cardetal2012, CitrinSides2008} and national security \citep{LahavCourtemanche2012}. According to the recent studies of public views on immigration, unfavorable attitudes toward immigrants and immigration are widespread and similar across receiving countries \citep{CitrinSides2008, Zicketal2008}.


There are two main schools of thoughts explaining public attitudes toward immigration: political economy and socio-psychological approaches \citep{DustmannPreston2007, HainmuellerHopkins2014}. Within the political economy literature which emphasizes material self-interest, one strand is based on the factor-proportions analysis model which focuses on the connection between individual skill levels and immigration-policy preferences \citep{ScheveSlaughter2001}. This model predicts that the inflow of immigrants affects the wages of the natives with similar skills and, therefore, natives are more likely to favor immigration if they are more skilled than immigrants and to oppose it otherwise \citep{HainmuellerHiscox2007, Mayda2006}. The other strand in the political economy literature focuses on the welfare systems and immigration's fiscal impacts \citep{Borjas1999}. It argues that immigrants receive a large share of the welfare and increase fiscal pressures to raise taxes. In this regard, natives with higher incomes are more supportive of restrictive immigration policy than those with lower incomes \citep{DustmannPreston2006}.


Instead of underscoring material self-interest, the other school of thoughts emphasizes the perceptions of sociotropic threats on the receiving country as a whole, including economic and cultural threats. Some research shows that personal economic situation matters little to public attitudes toward immigration, but the assessments of the impact of immigration on national economy and the general feelings about immigrants are important determinants of opinion formation \citep{Citrinetal1997}. This finding opens up an alternative avenue into the study of formulating opinions on immigration. That is, natives' opinion toward immigration is a result of the racial and ethnic stereotyping. For example, \cite{BurnsGimpel2000} shows that the fear of economic insecurity is related to people's negative stereotypes of racial and ethnic groups, which leads to a desire for restrictive immigration policy. 


Furthermore, symbolic predispositions, such as preferences for cultural unity and prejudice against reference groups, are influential on attitudes toward immigration \citep[e.g.,][]{SidesCitrin2007}. Under the broad framework of social comparison theories, individual attitudes and behavior can be influenced by community relationships. For one thing, the categorization of people causes in-group members to favor their own group and disapprove of the out-group \citep{ChandlerTsai2001}. For the other, the influx of the out-group changes the composition of the population and compositional amenities \citep{Cardetal2012}. It has been shown that the perceptions of immigration's cultural impacts on the receiving countries are prevalent in the United States and European countries \citep{CitrinSides2008, HainmuellerHiscox2007}. In this regard, individuals with higher levels of tolerance of out-group members and/or less perceived cultural threats are supportive of increased immigration. 



\subsection{Immigration Composition and Diverse Attitudes}


Most, if not all, of the studies of public attitudes toward immigration do not differentiate among immigrants and implicitly refer immigrants to blue-collar workers. This is because scholars focus more on immigration issues in economically developed countries, such as the United States, Canada, and Western European countries, than those in other countries and most immigrants in these countries are laborers from developing countries. Failing to account for the differences among immigrants may cause inferential problems. For example, the finding of the positive correlation between natives' skills and support for immigration may result from the premise that respondents think of low-skilled labor when answering survey questions about immigration policy \citep{ScheveSlaughter2001}.


To fill this gap, we extend the discussion to natives' attitudes toward both immigrant workers and professionals from the perspectives of political economy and socio-psychological approaches, respectively. According to the factor proportion model, which assumes perfect substitutability between natives and immigrants, the inflow of immigrants might lead to an decrease in the wages of similarly skilled natives \citep{ScheveSlaughter2001}. It has been shown that the natives tend to oppose the inflow of immigrants with similar skill level \citep{HansonScheveSlaughter2007}. The logic can be further applied to contrast two types of occupations---workers and professionals. As a result, native workers are likely to favor less restrictive policy for immigrant professionals but favor more restrictive policy for immigrant laborers. By the same token, native professionals are likely to support an increase of immigrant laborers while to oppose an influx of skilled immigrants. 


Moreover, concerning immigration's fiscal impacts, the literature shows that income is negatively correlated to supportive attitudes toward immigration \citep{DustmannPreston2006, DustmannPreston2007}. It is also shown that, however, the negative correlation between income and pro-immigration attitudes is found in countries where natives are more skilled than immigrants and that there is a positive correlation between individual skills and pro-immigration preferences \citep{FacchiniMayda2006}. These results suggest that the negative relationship between income and pro-immigration preferences found in the previous studies is likely due to the premise that respondents think of low-skilled laborers when answering survey questions about immigration policy. In this regard, low-skilled immigrations increases fiscal pressures to raise taxes while high-skilled immigration has the opposite effects. By differentiating between skill levels of immigrants, \cite{Hansonetal2007} shows that natives with higher incomes are more supportive of high-skilled immigrants and more opposed to low-skilled immigrants than those with lower incomes. In terms of occupations, we expect that natives with higher incomes are likely to favor immigration policy for immigrant professionals.


The socio-psychological approach discusses sociotropic threats on the receiving country and the general impression about immigrants perceived by the natives. It argues that the natives are concerned about the difficulties that the immigrants fit in with the culture in the receiving countries and the social problems caused by the immigrants \citep[e.g.,][]{Snidermanetal2004}. In this regard, the natives should have the same concerns with both workers and professionals. That is to say, in terms of cultural unity, the natives are opposed to the influx of immigration regardless of immigrants' profession.


The natives, however, usually differentiate among particular immigrant groups based on certain content of stereotypes such as nationality, race, and ethnicity \citep{LeeFiske2006}. The stereotypes of immigrants further influences information processing \citep{Vinacke1957} and the formation of prejudice \citep{BodenhausenWyer1985}, which makes some particular immigrant groups be more preferred to others. For example, Swiss voters are more likely to be opposed to the inflow of immigrants from Turkey and Yugoslavia than that from elsewhere in Europe \citep{HainmuellerHangartner2013}. In the Britain, by the same token, the respondents prefer white immigrant groups to non-white ones \citep{Ford2011}.


From the perspective of socio-psychological approach, public attitudes toward immigrant workers and immigrant professionals are shaped by stereotyping along with nationality, race, and ethnicity. The general perception of immigrants as low competence is due to the fact that most of the immigrants are less-skilled workers who are from developing countries, non-white, or non-Anglo-Saxons \citep{LeeFiske2006, ScheveSlaughter2001}. The stereotypes of immigrants further trigger the formation of negative emotions---in particular, anxiety---and shape anti-immigration attitudes \citep{BraderValentinoSuhay2008}. In the United States, for example, people with negative stereotypes of Latinos or Asian are more in favor of restrictive immigration policy \citep{ChandlerTsai2001}. Unlike the perceptions of laborers, people usually have a positive impression on immigrant professionals such as intelligence, gentleness, neatness, good manners, etc. These features of professionals usually attach to a reflection of white, Anglo-Saxons from economically developed countries. In this regard, the natives are less likely to oppose the inflow of foreign-born professionals.


In sum, we have several predictions according to the above discussions. Concerning labor market competition, first of all, the factor proportion model suggests that native workers are opposed to immigrant workers while native professionals favor restrictive policy for skilled immigrants. Second, the strand within the political economy approach that focuses on fiscal impacts of immigrants states that natives with higher incomes are more supportive of immigrant professionals and more opposed to immigrants workers than their poorer counterparts. Third, considering sociotropic effects of immigrants, the natives with preferences for cultural unity are equally opposed to immigrant professionals and immigrant workers. Finally, natives with group-specific stereotypes in mind have different levels of tolerance toward immigrant professionals and immigrant workers.



%Symbolic predispositions, such as preferences for cultural unity, have a stronger statistical effect than economic dissatisfaction \citep{SidesCitrin2007}

%skilled individuals and unskilled labor/immigrants \citep{Mayda2006}

%in Europe, people with higher levels of education and occupation skills are more likely to favor immigration \citep{HainmuellerHiscox2007}

%not all immigrant-related attitudes have an explicit ethno-racial component \citep{CeobanuEscandell2010}

%the caution of using multi-item measures, which might not differentiate between within- and between-country variation \citep{CeobanuEscandell2010}

%one of the suggestions for future research in the field of immigration is the immigrant composition \citep{CeobanuEscandell2010}




%\subsection{Government Performance on Immigration Issues}


%citizenship regimes can be understood as institutionalized systems of formal and informal norms that define access to membership, as well as rights and duties associated with membership \citep{Weldon2006}




\section{Case Selection, Data, and Measurements}


In this section, we discuss the case of Taiwan---a newly consolidated democracy and economically developing country in East Asia. Taiwan is a largely mono-ethnic country where native-born residents have strong stereotypical thinking on the connections between the images of ethnicities, countries of origin, and occupations. This feature makes Taiwan an appropriate case for us to test the predictions developed from different theoretical approaches.


\subsection{The Case of Taiwan}


Most of previous studies of immigration attitudes focus on economically developed countries such as the United States \citep[e.g.,][]{Citrinetal1997, ScheveSlaughter2001}, European countries \citep[e.g.,][]{DustmannPreston2007, HainmuellerHiscox2007}, and Japan \citep{Green2017}. In these countries, immigrants with different skill levels or occupations probably have the same countries and, therefore, the types of occupation are not strongly correlated to the stereotypes of immigrants' countries of origin, races, or ethnicities. For developing countries such as Taiwan, in contrasts, the differences between professionals and workers are obvious, in particular the overlaps over the images of races, countries of origin, or occupations.


In Taiwan, public perceptions of citizenship are strongly linked to ethnicity, which means that an immigrant can be easily recognized by their appearance, for two reasons. First, Taiwan is one of the countries that adopt \emph{jus sanguinis}, in which a person's citizenship is determined by their parents. Second, Taiwan has strict immigration regulations, by which immigrants hardly get residency and citizenship. Generally speaking, there are three types of migration in Taiwan: marriage migration, immigrant laborers, and white-collar foreign professionals. Foreigners in Taiwan can apply for citizenship or permanent residency if they marry to Taiwanese. Otherwise, only managers, investors, and professionals can apply for permanent residency. In the future, mid-level technical personnel and caretakers can apply for permanent residency after working for seven consecutive years\footnote{The Legislative Yuan passed the Act for the Recruitment and Employment of Foreign Professionals on November 22, 2017. The Executive Yuan approved New Economic Immigration Law on May 15, 2018. The former eases policies regarding certification, residency, health insurance, tax, and pensions for foreign professionals and the letter broaden the scope by including mid-level technical workers in the industrial and social care sectors.}.


In recent years, the transformation of demographic and economic structure resulted in the unbalanced labor supply and demand in Taiwan. On the one hand, the birth rate has dropped from 2.297\% in 1981 to 0.823\% in 2017. Population size increased by 327,947 people in 1981, but it only increased by 31,411 in 2017. On the other hand, the economic structure gradually changed from agriculture to production and service industries. The number of people who worked in production and service industries increased from 2.3 million in 1971 to 8 million in 2011. The population transition creates demand for labors, while the economic development requires more investment and educated professionals.


In October 1989, the Council of Labor Affairs (MOL) decided to open up Taiwan's job market to immigrant workers to respond to these issues. Foreign blue-collar workers from Thailand, Philippines, Indonesia, Malaysia, Vietnam, China and Mongolia are now allowed to work in Taiwan as part of solution of labor shortages. According to MOL, as of the end of December 2017, there were 676,142 foreign workers in Taiwan. Among them, 60.43\% employed as manufacturing workers, 0.76\% as construction workers, 1.82\% on fishing crews, 36.71\% as caretakers and 0.29\% as domestic helpers.\footnote{These employees can work as long as 12 years.}


About the same time, marital immigrants entered into Taiwan in a large scale. One of the major reasons is young male in the rural area lack opportunities of knowing young female as the service industry attracts many female labors. Up to the end of 2017, it is estimated that about 350 thousand of couples whose spouses are from China and 176 thousand from other countries. (Gender Equality Committee of the Executive Yuan). Among them, 40 percent of Chinese spouses get residence permits and 68 percent of international spouse are granted national identity cards.\footnote{According to Cross-Strait People's Relations Ordinance, Chinese spouses can become naturalized after living in Taiwan for six years. Other countries' spouses also need six years to become citizens, and they have to renunciate their nationality afterwards.} There are mixed evidences about the birth rate of foreign spouses \citep{MoLai2004, Yangetal2012}. Many foreign spouses are working full-time or part-time as caretakers, waitress, and domestic helpers.


Although migrant workers and marital immigrants may attenuate the problem of labor supply as the birth rate is getting lower, brain drain becomes a serious problem for the economy that needs to be re-configured. In 2015, it is estimated that 724,000 people worked overseas, which is about 10\% higher than 2009. Stagnant salary is often cited as the major reason of emigration. \cite{LinChangLu2017} analyzed data from 1980 to 2012 and pointed out that the the gross domestic product per capita (GDP) is negatively correlated with real wage in Taiwan; labors have limited share of economic growth. Despite that, Taiwan's government has aggressively attracted foreign talents. According to MOL, the number of foreign professionals increased from 11,228 in 2004 to 18,412 in 2017. The remarkable gap between the scale of immigration and emigration pushes the government to amend the Act for the Recruitment and Employment of Foreign Professionals, which is passed by the Legislative Yuan in 2017. The new law allows foreign professionals to extend their work permission from 3 years to 5 years. Foreigners can apply for temporary visa to find a job. Certainly, it remains to be seen whether more foreign professionals are coming to work in Taiwan.


The abuse or mistreat of migrant workers is often made local headlines. They include lack of freedom to change employer, live-in situations with employers, and high broker fees in mother countries (Jeremy Oliver, 2017). In 2016, the government amended the Employment Service Act to improve the structures. After being hired, migrant workers do not have to leave the country every three years, which reduces the brokerage fees for re-entering Taiwan. The employers also cannot refuse to give them paid leave as it is required by Labor Standards Act, a statue that mainly regulates domestic labor and employer. According to a survey conducted by MOL, the average salary of foreign manufacturing workers and caretakers have been increasing for the past three years. (Ya-wen Li, 2018)


Like other countries, Taiwanese are concerned that migrant workers replace vanishing Taiwanese workers. Lin (2001) investigated the geographical distribution of migrant workers and concluded that it is difficult for domestic workers to move to the area that are populated by migrant workers. \cite{TsayLin2001} found that foreign labor has a negative effect for the semi- and less-skilled construction workers but benefits employment for managerial/professional workers. Lan (2010) found that employers of migrant workers tend to agree that migrant work forces can supplement local workers but employers of local workers tend to have a perception that local workers are replaced by migrant labors. In sum, the previous literature has mixed evidence about the influence of foreign labors in terms of employment.


Human trafficking is another pressing challenge for Taiwan. The victims are girls and women from Vietnam, Thailand, and China. \cite{Huang2017} examined the characteristics of victims in 132 court proceedings related to the Human Trafficking Prevention Act (HTPA) enacted in 2009. She found that about half of the cases (56.8\%) involve adult foreign victims. Lured by higher wages and not allowed to switch employers, some migrant workers choose sex industry (Liu, 2015). MOL estimated that the number of missing foreign workers has declined over the years and reached at 52,000 up until 2017(Central News Agency, 2018).


In brief, Taiwan government well managed migrant workers regarding their salary and impact on domestic job market. But the abuse and mistreat of migrant workers remains a serious issue. They range from age exploitation, verbal abuse to physical violence. The issue of human trafficking also needs to be addressed by law enforcement and prosecution. As the government intends to make the society more inclusive and diverse, it is important to examine public opinion toward the immigration.



\begin{table}[htp!]
%\vspace{20pt}
\begin{center}
  \begin{threeparttable}
\caption{Public Attitudes toward Immigrants in Taiwan, 2016}
\label{table1}
\newcolumntype{W}{D{.}{.}{5}}
\begin{tabular}{lWWWWW} 
\toprule
\multicolumn{1}{l}{} &  \multicolumn{1}{l}{Strongly} & \multicolumn{1}{l}{} & \multicolumn{1}{l}{} & \multicolumn{1}{l}{} & \multicolumn{1}{l}{Strongly}     \\
\multicolumn{1}{l}{Questions} &  \multicolumn{1}{l}{Disagree} & \multicolumn{1}{l}{Disagree} & \multicolumn{1}{l}{Neither} & \multicolumn{1}{l}{Agree} & \multicolumn{1}{l}{Agree}     \\
\midrule 
Immigrants are harmful to culture  & 4.46 & 63.06 & 7.87 & 21.72  &  2.89 \\
Immigrants are good for economy  & 3.44 & 41.27 & 11.71 & 41.93 & 1.65   \\
\bottomrule
\end{tabular}
\begin{tablenotes}
\item \footnotesize{Note: Data are weighted to reflect the characteristics of the national population and presented in percentage.}
\end{tablenotes}
  \end{threeparttable}
\end{center}
\end{table} 



\textcolor{red}{As to the general public's attitude toward immigrants, it is somewhat ambivalent. In a survey conducted in the early 2016 asking that ``Our country's culture is generally harmed by immigrants,'' two-thirds (67.52\%) of respondents disagreed or strongly disagreed \citep{Huang2016} (see Table 1). This indicates that most people in Taiwan are tolerant to immigrants in culture. However, in response to the statement ``Immigrants are generally good for our country's economy,'' respondents were highly polarized. With the exception that 11.71\% neither agreed or disagreed with the statement, all others were evenly divided into agree group and disagree group with around 44\% each. This means that people in Taiwan hardly had consensus toward immigrants on economic dimension. It is interesting to further decompose immigrants into white and blue color workers and then examine if the general public’s attitudes vary toward types of immigrants.}



\subsection{Data Description and Measures}


We analyze survey data  in Taiwan with a nationally representative sample of 1966 respondents conducted by the Taiwan Social Change Survey Project \citep{Fu2017}. The survey data were collected by face-to-face interview from August 7th to November 27th in 2016, \textcolor{red}{which is a part of the seventh round of Taiwan Social Change Survey Project from 2015 to 2019. The theme of the survey in 2016 is \emph{Citizens and the Role of Government}.} In addition to a series of questions about background demographic, the questionnaire also covers questions about what government should do on immigration.\footnote{The questions are provided in Appendix A.} We first evaluate public attitudes toward foreign professionals and laborers. In Table 2, as can be seen, a majority of the respondents (77.16\%) think that the Taiwan government should allow foreign professionals to become citizens of Taiwan while only 16.94\% of the respondents think that the Taiwan government should not do so. In contrast, about a half of the respondents (47.45\%) agree that the government should allow foreign laborers to become citizens of Taiwan and a half of the respondents (47.92\%) think that the Taiwan government should not do so. The distributions in Table 2 suggest that Taiwanese people welcome foreign professionals more than foreign laborers.



\begin{table}[htp!]
%\vspace{20pt}
\begin{center}
  \begin{threeparttable}
\caption{Public Attitudes toward Granting Citizenship to Immigrants in Taiwan, 2016}
\label{table2}
\newcolumntype{W}{D{.}{.}{5}}
\begin{tabular}{lWWWWW} 
\toprule
\multicolumn{1}{l}{Questions} &  \multicolumn{1}{l}{Definitely not} & \multicolumn{1}{l}{Probably not} & \multicolumn{1}{l}{Probably} & \multicolumn{1}{l}{Definitely} & \multicolumn{1}{c}{Missing}    \\
\midrule 
Professionals being citizens  & 5.65 & 11.29 & 42.37 & 34.79 & 5.90   \\
Workers being citizens  & 17.60 & 30.32 & 32.60 & 14.85 & 4.63    \\
\bottomrule
\end{tabular}
\begin{tablenotes}
\item \footnotesize{Note: Data are weighted to reflect the characteristics of the national population and presented in percentage.}
\end{tablenotes}
  \end{threeparttable}
\end{center}
\end{table} 





\begin{table}[ht!]
%\vspace{20pt}
\begin{center}
  \begin{threeparttable}
\caption{Cross-tabulation of Attitudes toward Two Types of Immigrants}
\label{table3}
\newcolumntype{W}{D{.}{.}{9}}
\begin{tabular}{lWWWW} 
\toprule
\multicolumn{1}{l}{Allow foreign professionals} &  \multicolumn{4}{c}{Allow foreign workers to become citizens}     \\
\multicolumn{1}{l}{to become citizens} &  \multicolumn{1}{l}{Definitely not} & \multicolumn{1}{l}{Probably not} & \multicolumn{1}{l}{Probably} & \multicolumn{1}{l}{Definitely}     \\
\midrule 
Definitely not  & 80.18 (89) & 11.71 (13) & 6.31 (7) & 1.80 (2)    \\
Probably not  & 20.72 (46) & 65.77 (146) & 11.26 (25) & 2.25 (5)   \\
Probably & 13.12 (108) & 33.78 (278) & 49.70 (409) & 3.40 (28)     \\
Definitely & 13.39 (90) & 20.39 (137) & 28.27 (190) & 37.95 (255)     \\
\bottomrule
\end{tabular}
\begin{tablenotes}
\item \footnotesize{Note: Data are weighted to reflect the characteristics of the national population; row percentages are presented and frequencies are in the parentheses; 138 samples are missing.}
\end{tablenotes}
  \end{threeparttable}
\end{center}
\end{table} 



\textcolor{red}{Table 3 displays the cross-tabulation of public attitudes toward two types of immigrants. The distribution implies a positive correlation between the public attitudes toward foreign professionals and foreign laborers. More specifically, Spearman's rank correlation coefficient is 0.45 suggesting that the two variables are moderately positively correlated to each other. Therefore, we have to take the positive correlation into account when we examine the determinants of either variable.}




\begin{table}[ht!]
%\vspace{20pt}
\begin{center}
  \begin{threeparttable}
\caption{Cross-tabulation of Occupations and Attitudes toward Immigration}
\label{table4}
\newcolumntype{W}{D{.}{.}{9}}
\begin{tabular}{lWWWW} 
\toprule
\multicolumn{1}{l}{} &  \multicolumn{4}{c}{Allow foreign professionals to become the citizens}     \\
\multicolumn{1}{l}{Occupations} &  \multicolumn{1}{l}{Definitely not} & \multicolumn{1}{l}{Probably not} & \multicolumn{1}{l}{Probably} & \multicolumn{1}{l}{Definitely}     \\
\midrule 
Professionals  & 3.24 (19) & 7.51 (44) & 44.37 (260) & 44.88 (263)    \\
Skilled-workers  & 5.88 (42) & 14.15 (101) & 46.08 (329) & 33.89 (242)   \\
Low-skilled workers & 9.87 (47) & 13.24 (63) & 43.49 (207) & 33.40 (159)     \\
Armed forces & 0.00 (0) & 16.67 (4) & 33.33 (8) & 50.00 (12)     \\
\midrule 
\midrule 
\multicolumn{1}{l}{} &  \multicolumn{4}{c}{Allow foreign workers to become the citizens}     \\
\multicolumn{1}{l}{Occupations} &  \multicolumn{1}{l}{Definitely not} & \multicolumn{1}{l}{Probably not} & \multicolumn{1}{l}{Probably} & \multicolumn{1}{l}{Definitely}     \\
\midrule 
Professionals  & 13.45 (78) & 30.17 (175) & 38.10 (221) & 18.28 (106)    \\
Skilled-workers  & 18.77 (137) & 33.56 (245) & 33.15 (242) & 14.52 (106)   \\
Low-skilled workers & 23.67 (116) & 30.20 (148) & 31.22 (153) & 14.90 (73)     \\
Armed forces & 9.09 (2) & 22.73 (5) & 54.55 (12) & 13.64 (3)     \\
\bottomrule
\end{tabular}
\begin{tablenotes}
\item \footnotesize{Note: Data are weighted to reflect the characteristics of the national population; row percentages are presented and frequencies are in the parentheses; 166 and 144 samples are missing, respectively.}
\end{tablenotes}
  \end{threeparttable}
\end{center}
\end{table} 



\begin{table}[ht!]
%\vspace{-10pt}
\begin{center}
  \begin{threeparttable}
\caption{Cross-tabulation of Income and Attitudes toward Immigration}
\label{table5}
\newcolumntype{W}{D{.}{.}{9}}
\begin{tabular}{lWWWW} 
\toprule
\multicolumn{1}{l}{Income} &  \multicolumn{4}{c}{Allow foreign professionals to become the citizens}     \\
\multicolumn{1}{l}{(US dollar)} &  \multicolumn{1}{l}{Definitely not} & \multicolumn{1}{l}{Probably not} & \multicolumn{1}{l}{Probably} & \multicolumn{1}{l}{Definitely}     \\
\midrule 
1000 and below  & 7.65 (72) & 15.94 (150) & 45.48 (428) & 30.92 (291)    \\
1000-2000  & 4.40 (27) & 9.14 (56) & 46.49 (285) & 39.97 (245)   \\
2000-3000 & 0.78 (1) & 3.12 (4) & 42.97 (55) & 53.12 (68)     \\
3000-4000 & 0.00 (0) & 4.17 (2) & 45.83 (22) & 50.00 (24)     \\
4000 and above & 6.06 (2) & 0.00 (0) & 33.33 (11) & 60.61 (20)     \\
\midrule 
\midrule 
\multicolumn{1}{l}{Income} &  \multicolumn{4}{c}{Allow foreign workers to become the citizens}     \\
\multicolumn{1}{l}{(US dollar)} &  \multicolumn{1}{l}{Definitely not} & \multicolumn{1}{l}{Probably not} & \multicolumn{1}{l}{Probably} & \multicolumn{1}{l}{Definitely}     \\
\midrule 
1000 and below  & 20.68 (200) & 32.26 (312) & 33.30 (322) & 13.75 (133)    \\
1000-2000  & 14.78 (90) & 33.17 (202) & 34.98 (213) & 17.08 (104)   \\
2000-3000 & 11.72 (15) & 26.56 (34) & 39.06 (50) & 22.66 (29)     \\
3000-4000 & 10.64 (5) & 27.66 (13) & 44.68 (21) & 17.02 (8)     \\
4000 and above & 21.21 (7) & 18.18 (6) & 39.39 (13) & 21.21 (7)     \\
\bottomrule
\end{tabular}
\begin{tablenotes}
\item \footnotesize{Note: Data are weighted to reflect the characteristics of the national population; row percentages are presented and frequencies are in the parentheses; 203 and 182 samples are missing, respectively.}
\end{tablenotes}
  \end{threeparttable}
\end{center}
\end{table} 





\begin{table}[ht!]
%\vspace{20pt}
\begin{center}
  \begin{threeparttable}
\caption{Cross-tabulation of Attitudes toward Transnational Marriage and Immigration}
\label{table6}
\newcolumntype{W}{D{.}{.}{9}}
\begin{tabular}{lWWWW} 
\toprule
\multicolumn{1}{l}{Transnational} &  \multicolumn{4}{c}{Allow foreign professionals to become the citizens}     \\
\multicolumn{1}{l}{marriage} &  \multicolumn{1}{l}{Definitely not} & \multicolumn{1}{l}{Probably not} & \multicolumn{1}{l}{Probably} & \multicolumn{1}{l}{Definitely}     \\
\midrule 
Definitely not  & 30.56 (44) & 12.50 (18) & 22.92 (33) & 34.03 (49)    \\
Probably not  & 5.39 (32) & 20.37 (121) & 48.15 (286) & 26.09 (155)   \\
Probably & 2.27 (17) & 8.13 (61) & 55.87 (419) & 33.73 (253)     \\
Definitely & 4.14 (11) & 4.51 (12) & 17.67 (47) & 73.68 (196)     \\
\midrule 
\midrule 
\multicolumn{1}{l}{Transnational} &  \multicolumn{4}{c}{Allow foreign workers to become the citizens}     \\
\multicolumn{1}{l}{marriage} &  \multicolumn{1}{l}{Definitely not} & \multicolumn{1}{l}{Probably not} & \multicolumn{1}{l}{Probably} & \multicolumn{1}{l}{Definitely}     \\
\midrule 
Definitely not  & 52.45 (75) & 17.48 (25) & 14.69 (21) & 15.38 (22)    \\
Probably not  & 18.30 (110) & 42.26 (254) & 29.12 (175) & 10.32 (62)   \\
Probably & 12.70 (96) & 28.97 (219) & 45.37 (343) & 12.96 (98)     \\
Definitely & 15.71 (41) & 22.22 (58) & 25.67 (67) & 36.40 (95)     \\
\bottomrule
\end{tabular}
\begin{tablenotes}
\item \footnotesize{Note: Data are weighted to reflect the characteristics of the national population; row percentages are presented and frequencies are in the parentheses; 212 and 205 samples are missing, respectively.}
\end{tablenotes}
  \end{threeparttable}
\end{center}
\end{table} 




We then provide cross-tabulations to show the relationship between economic/cultural factors and attitudes toward immigration. According to the discussion above, people's socioeconomic status such as occupations and income influences how they view immigration. Regarding the classification of occupations, we regroup the respondents into four categories based on international standard classification of occupations (ISCO-08): professionals, skilled workers, low-skilled workers, and armed forces occupations.\footnote{There are ten major groups in ISCO-08 provided by International Labor Organization: (1) managers; (2) professionals; (3) technicians and associate professionals; (4) clerical support workers; (5) services and sales workers; (6) skilled agricultural, forestry and fishery workers; (7) craft and related trades workers; (8) plant and machine operators and assemblers; (9) elementary occupations; (0) armed forces occupations. The classification of occupations in this article is as follows: professionals include (1) and (2); skilled workers include (3), (6), (7), and (8); low-skilled workers include (4), (5), and (9); armed forces occupations.} Table 2 displays the distributions of public attitudes toward immigration across occupations. In Table 2, we do not observe different attitudes toward professionals or workers among different types of occupations. The result in Table 2 shows the same pattern with that in Table 1, that is, Taiwanese people welcome foreign professionals more than foreign workers regardless of immigrants' occupations.


The relationship between income and public attitudes toward immigration is presented in Table 3. In Table 3, overall, a majority of respondents are in favor of granting citizenship to foreign professionals in each income level. People with higher incomes, however, are more willing to allow foreign professionals to become citizens than their poorer counterparts. In contrast, this pattern is not obvious in public attitudes toward immigrant workers. In other words, it seems that income is not related to people's attitudes toward immigrant workers.


Besides socioeconomic status, people's tolerance of different cultures also determines their attitudes toward immigration. In specific, people with greater cultural tolerance are more willing to allow foreigners to become citizens. To measure tolerance, we use the question asking respondents' attitudes toward transnational marriage. In Table 4, we display the cross-tabulation of public attitudes toward transnational marriage and immigration. We find that, as a whole, people who accept transnational marriage also allow both foreign professionals and foreign workers to be granted citizenship. People who do not favor transnational marriage, however, are more willing to allow foreign professionals rather than foreign workers to become citizens.




\section{Empirical Analysis of Individual Attitudes toward Immigrants}


We examine public attitudes toward immigrants in Taiwan by analyzing the survey data displayed above. In the following, we first introduce the statistical model used in the data analysis and then present the results of analysis.


\subsection{The Bayesian Bivariate Ordered Probit Model}


A bivariate ordered probit model (BOPM) is applied to analyzing the survey data, which is useful for modeling the correlation between two ordered response variables \citep[p.~227]{GreeneHensher2010}.\footnote{Bivariate ordered probit models can be considered as an extension of bivariate probit models, where the two response variables are binary.} We can derive the bivariate ordered probit model from the latent variable model. Suppose that, for the two response variables $y_{i,1}$ and $y_{i,2}$, respondent $i=1,\cdots,N$ provides a set of response ($y_{i,1}=j, y_{i,2}=k$) for $j=1,2,\cdots,J$ and $k=1,2,\cdots,K$ based on unobserved, latent traits $y_{i,1}^{*}$ and $y_{i,2}^{*}$ and threshold parameters $\tau_{1,j}$ and $\tau_{2,j}$. The two latent variables can be represented by:
\begin{empheq}[left={\empheqlbrace}]{align}
    y_{i,1}^{*} &= \bm{x}'\bm{\beta}+\varepsilon_{i,1}  \label{positive}\\
    y_{i,2}^{*} &= \bm{z}'\bm{\gamma}+\varepsilon_{i,2},  \label{negative}
\end{empheq}
where $\bm{x}=(x_{i,1}, x_{i,2}, \cdots, x_{i,M})$ and $\bm{z}=(z_{i,1}, z_{i,2}, \cdots, z_{i,G})$ are $M$-variate and $G$-variate predictors, respectively, $\bm{\beta}\in \mathbb{R}^{M}$ and $\bm{\gamma}\in \mathbb{R}^{G}$ are corresponding unknown parameter vectors, and $\bm{\varepsilon_{i}}=(\varepsilon_{i,1}, \varepsilon_{i,2})$ is the error term. The predictors in $\bm{x}$ and $\bm{z}$ can be the same or different \citep{Greene2012}. Given $y_{i,1}^{*}$, $y_{i,2}^{*}$, $\tau_{1,j}$, and $\tau_{2,j}$, we observe the responses as follows:
\begin{empheq}[left={\empheqlbrace}]{align}
    y_{i,1} &= j \text{ if } \tau_{1,j-1} < y_{i,1}^{*} \leq \tau_{1,j}   \\
    y_{i,2} &= k \text{ if } \tau_{2,k-1} < y_{i,2}^{*} \leq \tau_{2,k} 
\end{empheq}
where it is assumed that $\tau_{1,0}=\tau_{2,0}=-\infty$.


To relax the assumption that the two latent variables are independent, the error term $\bm{\varepsilon_{i}}$ is assumed to follow a bivariate standard normal distribution as follows:
\begin{align}
  \bpm \varepsilon_{i,1} \\ \varepsilon_{i,2}  \epm \sim \text{N}_{2} \left[ \bpm 0 \\ 0  \epm, \bpm 1 & \rho \\ \rho & 1 \epm \right].
\end{align} 
The correlation parameter $\rho$ captures the association between the two response variables.


We take a Bayesian approach to constructing the proposed bivariate probit model, so we complete the model specification by defining the prior distributions. We use uninformative prior distributions for the unknown parameters as follows:
\begin{align}
  \beta_{m} &\sim \text{N}(0, 100)\, \quad \text{for} \quad m=1,\cdots ,M, \\
  \gamma_{g} &\sim \text{N}(0, 100), \quad \text{for} \quad g=1,\cdots ,G, \\
  \rho &\sim \text{Uniform}(-1,1).
\end{align} 



\subsection{Results of Analysis}


We apply the Bayesian bivariate ordered probit model illustrated above to analyzing the survey data in Taiwan. In addition to our main explanatory variables---occupation, income, tolerance, we also include several control variables in the model. First, we include variables about people's evaluations of national economy and their own household's economic condition. Second, \textit{Social Status} measures self-evaluations of social rank, ranging from 1 to 10 with 1 as the bottom and 10 as the top. Third, \textit{Unemployment} represents respondents' employment status. Fourth, we include variables to represent five levels of education with illiteracy as the reference. Finally, \textit{Solving Problems} asks respondents about how successful the government in Taiwan deals with problems regarding foreign laborers. 



\begin{center}
%\vspace{10pt}
\begingroup
\renewcommand\arraystretch{0.68}
{\footnotesize
\begin{longtable}{l|cc}
\caption{Determinants of Public Attitudes in 2016}  \\
\hline
 &  \multicolumn{2}{c}{Attitudes toward}    \\
Explanatory Variable & Professionals  & Workers   \\
\hline
\endfirsthead
\multicolumn{3}{c}%
{\tablename\ \thetable: Determinants of Public Attitudes in 2016 (Continued)} \\
\hline
Outcome Variable &  \multicolumn{2}{c}{Attitudes toward}     \\
 & Professionals  & Workers   \\
\hline
\endhead
\hline \multicolumn{3}{r}{} \\
\endfoot
\hline
\hline \multicolumn{3}{l}{Data source: 2016 Taiwan Social Change Survey (Round 7, Year 2)} \\
          \multicolumn{3}{l}{Note: $90\%$ HPD intervals are presented.} \\
\endlastfoot
\textbf{Occupation} (Military=0)  &               &            \\ 
Professionals         & $-0.002$        & 0.134              \\
                       & $[-0.466, 0.483]$   & $[-0.397, 0.593]$       \\
Skilled Worker       & $-0.202$        & $-0.068$              \\
                     & $[-0.633, 0.223]$   & $[-0.571, 0.396]$       \\
Low-skilled Worker    & $-0.103$        & $-0.075$              \\
                     & $[-0.516, 0.338]$   & $[-0.593, 0.406]$       \\
\textbf{Country Eco.} (Same=0)   &               &            \\
Better         & $0.059$                & $0.326$               \\
            & $[-0.230, 0.342]$   & $[0.060, 0.585]$     \\
Worse   & 0.076                & $0.044$                  \\
           & $[-0.074, 0.221]$   & $[-0.091, 0.187]$    \\
\textbf{Household Eco.} (Same=0)   &               &            \\
Better         & $-0.116$                & $0.030$               \\
            & $[-0.329, 0.124]$   & $[-0.195, 0.251]$     \\
Worse   & $-0.085$                & $-0.238$                  \\
           & $[-0.252, 0.096]$   & $[-0.402, -0.071]$    \\
\textbf{Income} (below US\$1000=0)   &               &            \\
1000-2000         & $0.290$                & $0.072$               \\
            & $[0.111, 0.466]$   & $[-0.090, 0.243]$     \\
2000-3000         & $0.747$                & $0.333$               \\
            & $[0.431, 1.038]$   & $[0.047, 0.615]$     \\
3000-4000         & $0.713$                & $0.343$               \\
            & $[0.263, 1.137]$   & $[-0.094, 0.742]$     \\
above 4000         & $0.830$                & $0.206$               \\
            & $[0.288, 1.352]$   & $[-0.274, 0.719]$     \\
\textbf{Tolerance}      & $0.783$                & 0.588         \\
         & $[0.645, 0.928]$   & $[0.449, 0.730]$    \\ 
\textbf{Social Status}      & $-0.070$                & 0.007         \\
         & $[-0.117, -0.031]$   & $[-0.033, 0.047]$    \\ 
\textbf{Unemployment}      & $0.171$                & $-0.031$         \\
         & $[0.011, 0.344]$   & $[-0.189, 0.128]$    \\ 
\textbf{Education} (Illiteracy=0)       &               &           \\
Junior high     & $-0.025$                & $0.149$           \\
           & $[-16.681, 16.450]$   & $[-15.857, 16.463]$    \\  
Senior high     & $0.018$                & 0.115          \\
           & $[-16.449, 16.331]$   & $[-16.612, 16.028]$      \\  
College          & $-0.069$                & $-0.037$         \\
           & $[-16.114, 16.662]$   & $[-16.650, 15.903]$    \\  
University and above  & $0.008$        & $0.011$         \\
          & $[-17.360, 15.244]$   & $[-16.748, 16.324]$    \\  
\textbf{Solving Problems} (Unsuccessful=0)   &               &            \\
Neither         & ---                & $-0.239$               \\
            & ---   & $[-0.438, -0.048]$     \\
Successful   & ---                & $-0.195$                  \\
           & ---   & $[-0.315, -0.061]$    \\
\hline
\textbf{Cutpoint 1}  & $-2.356$                & $-1.131$          \\
          & $[-2.865, -1.808]$   & $[-1.657, -0.667]$       \\
\textbf{Cutpoint 2}  & $-1.272$                & 0.164          \\
          & $[-1.744, -0.718]$   & $[-0.354, 0.630]$       \\
\textbf{Cutpoint 3}  & $0.603$                & 1.715          \\
          & $[0.121, 1.148]$   & $[1.187, 2.187]$       \\
\hline
\bm{$\rho$}           & \multicolumn{2}{c}{0.957}              \\
                      & \multicolumn{2}{c}{[0.915, 0.997]} 
\label{table6}
\end{longtable}
}
\endgroup
\end{center}



The results of analysis are presented in Table 5.\footnote{The estimation was performed with three parallel chains of 100,000 iterations each to be conservative. The first half of the iterations were discarded as a burn-in period and 10 as thinning and thus 15,000 samples were generated.} Several findings are summarized as follows. First, we find no evidence that occupation is associated with attitudes toward immigration, including both foreign professionals and foreign laborers. Second, we find that people with higher incomes are more likely to allow foreign professionals to become citizens. But this is not the case for foreign workers. Third, the assessments of national and household economic condition is not correlated to attitudes toward foreign professionals but is partly related to attitudes toward foreign workers. In specific, people who think that Taiwan's economy has gotten better are more likely to allow foreign workers to become citizens; people who think that their own household economic condition has gotten worse are less likely to allow foreign workers to become citizens. 


Fourth, people who can accept transnational marriage are more likely to allow foreign professionals and foreign workers to become citizens. This result is consistent with the findings of previous studies in which people who have greater cultural tolerance are more likely to favor less restrictive immigration policy. Fifth, we do not find that education is correlated to attitudes toward immigration. The reason is that, in previous studies of public attitudes toward immigration, education is used to capture people's tolerance of cultural diversity, but we have a measure for cultural tolerance which absorbs the explanatory power of education. Sixth, people who think that the government successfully, or at least do not unsuccessfully, deal with problems regarding foreign laborers are less likely to support the inflow of immigrant workers. This result implies that people may think that granting citizenship to foreign workers is not necessary if the government provides them a good working environment. Finally, the two response variables are positively correlated to each other.



\section{Conclusion}


Public attitudes toward immigration are important in democratic countries in the sense that public opinion plays a role in shaping public policy. Since anti-immigration attitudes are observed across receiving countries, scholars provide economic and cultural explanations to the source of negative sentiments toward immigration and find evidence in countries such as the United States and Western European countries. What is missing, however, is that most of the previous studies focus on economically developed countries and ignore public attitudes toward immigration in developing countries.


To fill this gap, we analyze survey data in Taiwan---a largely mono-ethnic country where native-born residents have strong stereotypical thinking on the connections between the images of ethnicities, countries of origin, and occupations. According to the results of analysis, we find that, first, the natives do have different attitudes toward foreign professionals and foreign laborers. In general, Taiwanese people welcome foreign professionals more than foreign laborers. Second, people with higher incomes are more likely to allow foreign professionals. Third, people with positive assessment of national economy are more likely to allow granting citizenship to foreign laborers while those with negative assessment of household economy are less likely to allow granting citizenship to foreign laborers. Fourth, natives with greater cultural tolerance are more willing to grant citizenship to foreign professionals and foreign laborers. Finally, people who think that the government do not fail to deal with problems regarding foreign laborers are less likely to support the inflow of immigrant workers.





\clearpage


\newpage



\normalsize{
\singlespacing 
\bibliographystyle{/Users/ttsai/bib/apsr}
\bibliography{/Users/ttsai/bib/bibdata} 
}



\clearpage


\newpage



\appendix

\renewcommand{\thesection}{Appendix \Alph{section}}


\section{Question Wordings}

\setcounter{equation}{0}
\renewcommand{\theequation}{A-\arabic{equation}}


Below, we provide the prompts and questions that the respondents were asked about immigrants. We also provide the labels used in the main text of the article in parentheses. The choice-options provided for questions F3, F4 and F5 are: \textit{Definitely should, Probably should, Probably should not, and Definitely should not}.
\begin{quote}
Next, we are going to ask you some questions about fertility decline and immigration.
\end{quote}


\begin{table}[ht!]
%\vspace{20pt}
\begin{center}
  \begin{threeparttable}
\caption*{Table A1: Survey Questions about Immigration}
%\label{table1}
\begin{tabular}{l} 
\toprule
\midrule
F4. Do you think the Taiwan government should or should not allow foreign professionals to \\ 
\qquad become citizens of Taiwan, with the same rights and obligations (such as voting rights \\ 
\qquad and paying taxes) as we have? (Professionals being citizens)    \\
F5. Do you think the Taiwan government should or should not allow migrant laborers and care \\ 
\qquad workers to become citizens of Taiwan, with the same rights and obligations (such as voting \\ 
\qquad rights and paying taxes) as we have? (Workers being citizens)  \\
F3. Do you think the Taiwan government should or should not actively encourage Taiwanese  \\
\qquad to marry foreigners and live in Taiwan? (Transnational marriage)   \\
\toprule
\bottomrule
\end{tabular}
\begin{tablenotes}
  \item \footnotesize{Data source: 2016 Taiwan Social Change Survey (Round 7, Year 2).}
\end{tablenotes}
  \end{threeparttable}
\end{center}
\end{table} 


\clearpage


\newpage



\end{document}